\section*{Abstract}
\label{sec:abstract}

Reinforced random walks occur in many complex systems, everything from biological systems such as construction of blood vessels or neural networks to trail-laying ants can at some level be described by reinforced random walks. One particular type of reinforced random walk, the current-reinforced random walk, is implemented in \texttt{C++} with the library OpenMP for parallelism.

Results of simulating current-reinforced systems show that the characteristics of the constructed network and the time needed until a solution is found strongly depend on the path maintenance parameter, $\mu$, as well as the number of sources versus the number of sinks and their geographical placement. If in the linear state, i.e.\@ $\mu = 1$, the network will consist of the shortest paths from each source to each sink and it takes a long time to find a solution. If in the non-linear state, i.e.\@ $\mu > 1$, there is instead a preference to share paths rather than taking the shortest path and the time needed to find a solution is much smaller. Results also show that if the value of $\mu$ is increased enough the simulation breaks down and the solution found does not represent any network at all. 

The model is also tested on a graph over the street network of the Uppsala, Sweden, where source are placed in the residential areas and sinks are placed in the city center. This can be seen as a simulation of traffic flow during rush hours and used to understand how to deal with traffic congestion.

In order to improve the performance of the implementation, the nodes are ordered with the graph partitioning software METIS. Measurements of speedup and sizeup show that the parallel implementation significantly increase performance. 