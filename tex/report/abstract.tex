\section{Abstract}
\label{sec:abstract}

The occurrence of reinforced random walks are present in many complex systems, everything from biological systems such as construction of blood vessels or neural networks to trail-laying ants can at some level be described by a reinforced random. Modeling complex systems with reinforced random walks is often done by simulating particles traveling from node to node in a specified graph, constructing networks. The paths taken by the particles depend on local environmental parameters in each node. The parameters for a node can for example be the particle density within the node or the flow of particles to an other node. The probability for a particle to move to an other node then depend on these parameters.

In the particular underlying current-reinforced random walk for this paper the probability for a particle to move to an other node depends on the flow of particles to that node, i.e. the current of particles. The mathematical modeling is based on the research presented in \cite{Sumpter}. The basic idea of a current-reinforced random walk is to find the shortest path between a source and a sink for a given graph. This optimizes the path length for each individual path from the sources to the sinks. When utilizing non linear current-reinforced random walks a combination of path length and path maintenance is optimized. For example, ants may want to construct one big road rather than many small ones to easier keep needles and dirt of the road while not making each individual path that much longer.

The implementation of these current-reinforced random walks is done in \texttt{C++} with the library OpenMp for parallelism. In order to get better performance the nodes are numbered and ordered with graph partitioning schemes. This ensures better memory access and less communication between cores when running in parallel. Measurements of speedup and sizeup showed that the parallel implementation significantly increased performance. 