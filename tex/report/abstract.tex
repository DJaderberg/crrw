\section*{Abstract}
\label{sec:abstract}

The occurrence of reinforced random walks are present in many complex systems, everything from biological systems such as construction of blood vessels or neural networks to trail-laying ants can at some level be described by a reinforced random walk. Modelling complex systems with reinforced random walks is often done by simulating particles travelling from node to node in a specified graph, constructing networks where the paths taken by the particles depend on local environmental parameters in each node.

Results of simulating such systems show that the characteristics of the constructed network and the time needed until a solution is found strongly depend on the path maintenance parameter, $\mu$, as well as the number of sources versus the number of sinks and their geographical placement. If in the linear state, i.e.\@ $\mu = 1$, the network will consist of the shortest paths from each source to each sink and it takes a long time to find a solution. If in the non-linear state, i.e.\@ $\mu > 1$, there is instead a preference to share paths rather than taking the shortest path and the time needed to find a solution is much smaller. Results also show that if the value of $\mu$ is increased enough the simulation breaks down and the solution found does not represent any network at all. 

The model is also tested on a graph over the street network of the Uppsala, Sweden, where source are placed in the residential areas and sinks are placed in the city center. This can be seen as a simulation of traffic flow during rush hours and used to understand how to deal with traffic congestion.

The implementation is done in \texttt{C++} with the library OpenMP for parallelism. In order to get better performance the nodes are numbered and ordered with graph partitioning schemes. Measurements of speedup and sizeup show that the parallel implementation significantly increase performance. 